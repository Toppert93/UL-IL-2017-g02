\subsection{Operation Model for DistanceTo}

\label{OM-DistanceTo}


The \msrcode{DistanceTo} operation has the following properties:

	\begin{operationmodel}
	\addheading{Operation}
	\adddoublerow{DistanceTo}{used to determine how close two locations are. In the context of the iCrash system, we compute the distance between two GPS locations using the following Haversine formula.
	(more details can be found at: 
	  http://www.movable-type.co.uk/scripts/latlong.html
	  and http://www.gpsvisualizer.com/calculators\#distance
	)
	 }


	\addrowheading{Return type}
	\addsinglerow{ptReal}

		

	\addrowheading{Post-Condition (functional)}
	\addnumberedsinglerow{PostF}{The Haversine formula (ACOS(SIN(lat1)*SIN(lat2)+COS(lat1)*COS(lat2)*COS(lon2-lon1))*6371, 
	in which latitudes and longitudes are in radiant applied to the two dtGPS coordinates has to be greater than zero.}

	\end{operationmodel}



	% ------------------------------------------
	% MCL Listing
	% ------------------------------------------
	\vspace{1cm}
	The listing~\ref{OM-dtGPSLocation-DistanceTo-MCL-LST} provides the \msrmessir (MCL-oriented) specification of the operation.
	
	\scriptsize
	\vspace{0.5cm}
	\begin{lstlisting}[style=MessirStyle,firstnumber=auto,captionpos=b,caption={\msrmessir (MCL-oriented) specification of the operation \emph{DistanceTo}.},label=OM-dtGPSLocation-DistanceTo-MCL-LST]

	
	
	/* Post Functional:*/ 
	postF{let Result :ptReal in
	      	let EarthRadius : dtReal = 6371000 in 
	      	let Rlat1 : dtReal in let Rlat1a : dtReal in
	      	let Rlong1 : dtReal in let Rlong1a : dtReal in
	      	let Deltalat : dtReal in let Deltalong : dtReal in
	      	let a1: dtReal in let a2:dtReal in 
	      	let a3: dtReal in let a : dtReal in
	      	let c1: dtReal in let c: dtReal in
	      	
	      	and Rlat1 = self.latitude.toRad()
	      	and Rlat1a = adtValue.latitude.toRad()
	      	and Rlong1 = self.longitude.toRad()
	      	and Rlong1a = adtValue.latitude.toRad()
	      	and Deltalat = Rlat1a.sub(Rlat1)
	      	and Deltalong = Rlong1a.sub(Rlong1)
	      	and a1 = Deltalat.div(2).sin().power(2)
	      	and a2 = Rlat1.cos().mult(Rlat1a.cos())
	      	and a3 = Deltalong.div(2).sin().power(2)
	      	and a4 = a1.add(a2.mult(a3))
	      	and c1 = a4.sub(1).power(2)
	      	and c = atan(a4.power(2),c1).mult(2)
	      	and Result =  EarthRadius.mult(c).asptReal()}
	
	
	\end{lstlisting}
	\normalsize 
	
	
	
	





